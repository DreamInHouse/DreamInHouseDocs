% --- Document: Preamble ---
\documentclass[../main.tex]{subfiles}
\graphicspath{{\subfix{../assets/}}}
% --- Document: Start ---
\begin{document}
% Introduction
\section{Introdução}    
% Project Problens 
\subsection{Problema Encontrato}
Ao considerar que para adquirir ou anunciar determinado imóvel, deve-se ir a imobiliária, demandando custo, tempo, deslocamento, falta de descontração e dificuldade na captação de clientes. 

O custo vem da aquisição dos funcionários, dos recursos para a recepção de um cliente na imobiliária, falta de organização e muitas vezes perda de recursos importantes para o andamento do cliente. De acordo com o Tecimob, um site especializado na gestão de imobiliárias, grande parte dos problemas ocasionados são em relação a imobiliária não ter um sistema de gerenciamento dos dados ou de contribuição para a interação e comunicação de clientes e fornecedores, o que consequentemente gera danos às informações e insatisfação do cliente nos atendimentos. 

Obtemos à percepção que o tempo é algo valioso, e deve ser utilizado com sabedoria para alcançar êxito nas atividades diárias, influenciando negativamente quando não é utilizado de forma correta, sendo o que acontece quando o cliente precisa ir pessoalmente até o local para conseguir informações importantes de localização, tamanho, classificação, valor e entre outras informações relacionadas ao imovél, algo que poderia ser antecipado para melhor experiência. 

Deste modo, assim como à disponibilização de tempo pode gerar insatisfação ao consumidor, ele possuir à necessidade de se deslocar para ter conhecimento e consciência dos dados de uma propriedade imobiliária, bem como influencia de forma descontente.  

Observando o cenário, podemos analisar que cada parte do processo é conectada e importante para o sucesso de uma imobiliária, portando, à falta de descontração entre o cliente e um consultor é um ponto determinante para decidir se o consumidor vai continuar sua trilha na empresa ou não, pois nos alocamos em lugares pelo qual sentimos conforto, produzindo boas expectativas. 

Entretanto, um dos pontos essenciais nas empresas é a captação dos clientes, é como eles são conquistados através de uma boa imagem do local e bom atendimento, portanto, é indispensável na organização. 
% Project Organization
\subsection{Proposta para Solução}
Em uma primeira análise, deve-se ressaltar os principais objetivos do projeto, sendo a criação de uma rede social inovadora voltada diretamente para o nicho imobiliário, com o intuito de facilitar para que elas possam adquirir imóveis de forma permanente ou temporária e também anunciar, consistindo em uma propositura que coadjuva o trabalho das empresas. 

À Dream InHouse, uma rede social eficiente propõe e assegura para a solução do problema o bom desenvolvimento da economia dos custos e tempo, inclui o deslocamento como algo que pode ser necessário somente na finalização do acordo, a descontração dos clientes vai ser predominante em uma rede social, contendo várias possibilidades de interação e acompanhamento dos conteúdos, ocasionando a captação de clientes que estão sendo atendidos como clientes permanentes, e conquistando clientes novos. 
% Structure and Methods
\subsection{Objetivo do Projeto}
O objetivo desse projeto é a criação de uma rede social, à Dream InHouse, voltada justamente para o mercado imobiliário, com o intuito de realizar a conexão e interação entre pessoas, podendo elas adquirir imóveis de forma permanente ou temporária, anunciar ou se inserir no nicho somente para estudos e pesquisas.  

A plataforma como uma rede social facilita a comunicação entre os usuários, aproximando usuários que queiram adquirir um imóvel ou anunciar o seu, locadores e locatários, e até mesmo corretores, através de anúncios criativos e objetivos, além de fornecer uma incrível experiência de comunicação e interação entre os usuários, com uma ferramenta de classificação de imóveis e um chat para comunicação direta entre os usuários. 
% Responsibilities
\subsection{Público-Alvo}
O público-alvo da rede social pode ser tanto pessoas relacionadas ao ramo imobiliário, que possuam interesse em participar das interações, estar por dentro dos valores, categorias dos imóveis, melhores localizações, se o imóvel é bom ou não e entre outros. Contudo, pessoas que não são do nicho imobiliário também podem participar e estar inclusas nessas discussões, podendo assim estudar e adquirir conhecimento sobre a área.
% Responsibilities
\subsection{Divisão de Responsabilidades}
\subsubsection{Chrystian Mendes Franklin}
\paragraph{Responsabilidades:}
Responsável técnico pela implementação do sistema, pela implementação das tecnologias, pela coordenação do projeto, pelo desenvolvimento dos códigos do backend, pelos testes realizados em todas as 3 plataformas (Linux, Windows e MacOS).

\subsubsection{Mariana G. M. Morais}
\paragraph{Responsabilidades:}
Responsável técnica pela criação do design do sistema, pela criação das interfaces gráficas, pela implementação do frontend, pela implementação das tecnologias gráficas, pelo desenvolvimento dos códigos do frontend, pelos testes realizados em todas as 3 plataformas (Linux, Windows e MacOS), pela documentação do projeto, slides e diagramas.

\subsubsection{Richard M. C. Pinto}
\paragraph{Responsabilidades:}
Responsável técnico pela criação das interfaces gráficas, pela implementação do frontend, pela implementação das tecnologias gráficas, pelo desenvolvimento dos códigos do frontend, pelos testes realizados em todas as 3 plataformas (Linux, Windows e MacOS).

\subsubsection{Willians H. S. Silva}
\paragraph{Responsabilidades:}
Responsável técnico pela implementação das regras de negócio, pela implementação das tecnologias, pelo desenvolvimento dos códigos do backend, pelos testes realizados na plataforma MacOS, responsável pela documentação do projeto.

% --- Document: End ---
\end{document}

