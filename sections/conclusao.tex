% --- Document: Preamble ---
\documentclass[../main.tex]{subfiles}
\graphicspath{{\subfix{../assets/}}}
% --- Document: Start ---
\begin{document}
% conclusoes
\section{Conclusões}

\subsection{Principais Resultados Alcançados}
Neste projeto, desenvolvemos uma rede social voltada para o nicho do mercado imobiliário, seguindo a metodologia Waterfall. Os principais resultados alcançados incluem:

\begin{itemize}
    \item \textbf{Funcionalidades Implementadas:} Todas as funcionalidades planejadas foram implementadas com sucesso, incluindo cadastro e login de usuários, criação, edição e remoção de propriedades e postagens, recuperação de contas, e funcionalidades de chat entre usuários.
    \item \textbf{Interface Gráfica:} Desenvolvemos uma interface gráfica moderna e intuitiva, utilizando Swing, SwingX, MigLayout e Formdev FlatLaf.
    \item \textbf{Armazenamento em Memória:} Implementamos um sistema de armazenamento em memória, utilizando classes de armazenamento como um pseudo banco de dados.
    \item \textbf{Multiplataforma:} O sistema foi testado e validado em múltiplas plataformas (Linux, macOS, Windows), garantindo a compatibilidade e funcionamento consistente em diferentes ambientes.
\end{itemize}

\subsection{Lições Aprendidas}
Durante o desenvolvimento deste projeto, várias lições importantes foram aprendidas:

\begin{itemize}
    \item \textbf{Planejamento Detalhado:} A importância de um planejamento detalhado na metodologia Waterfall foi fundamental para o sucesso do projeto. Seguir rigorosamente as etapas de pesquisa, levantamento de requisitos, diagramação, prototipação, implementação, testes, implantação e manutenção garantiu a entrega de um produto final robusto.
    \item \textbf{Gestão de Dependências:} Utilizar ferramentas como Maven para a gestão de dependências facilitou significativamente o processo de desenvolvimento e garantiu a integridade do projeto.
    \item \textbf{Interface de Usuário:} O uso de bibliotecas avançadas para a interface de usuário (SwingX, MigLayout, FlatLaf) melhorou substancialmente a experiência do usuário, mostrando a importância de escolher as ferramentas certas para cada parte do projeto.
    \item \textbf{Armazenamento em Memória:} Implementar um pseudo banco de dados em memória foi uma experiência valiosa, destacando tanto as vantagens quanto as limitações desta abordagem.
\end{itemize}

\subsection{Recomendações para Trabalhos Futuros}
Com base na experiência adquirida durante este projeto, as seguintes recomendações são sugeridas para trabalhos futuros:

\begin{itemize}
    \item \textbf{Integração com Banco de Dados:} Para projetos futuros, considerar a integração com um banco de dados relacional ou NoSQL para armazenamento persistente de dados, aumentando a escalabilidade e robustez do sistema.
    \item \textbf{Adaptação para Web e Mobile:} Expandir o sistema para plataformas web e mobile, utilizando frameworks como Spring Boot para backend e tecnologias como React Native ou Flutter para frontend.
    \item \textbf{Melhoria na Segurança:} Implementar melhores práticas de segurança, incluindo criptografia de dados sensíveis, autenticação e autorização robustas, e proteção contra ataques comuns (e.g., SQL Injection, XSS).
    \item \textbf{Automatização de Testes:} Adotar práticas de desenvolvimento orientado a testes (TDD) e integração contínua (CI) para melhorar a qualidade e a manutenção do código.
    \item \textbf{Feedback do Usuário:} Implementar mecanismos de coleta de feedback dos usuários para orientar futuras melhorias e adaptações do sistema.
\end{itemize}

% --- Document: End ---
\end{document}

