% --- Document: Preamble ---
\documentclass[../main.tex]{subfiles}
\graphicspath{{\subfix{../assets/}}}
% --- Document: Start ---
\begin{document}
%implementation
\section{Implementação do Sistema}
% linguagem de Programação
\subsection{Linguagem de Programação}
\textbf{Java} (Oracle JDK 17) foi a linguagem de programação escolhida para o desenvolvimento do projeto.
A escolha foi feita com base na experiência da equipe de desenvolvimento e na compatibilidade 
com as bibliotecas necessárias para o projeto.
% Ferramentas
\subsection{Ferramentas de Desenvolvimento}
    \subsubsection{Como ferramentas auxiliares ao desenvolvimento utilizamos :}
    Para comunicação interna entre a equipe, o Whatsapp e o Discord.
    \subsubsection{Para Organização das etapas do projeto, utilizamos:}
    o Trello, o Google Docs, o World, o Excel, o Numbers, o Pages e o MS-Project.
    \subsubsection{Para Diagramação, utilizamos: }
    Draw.io.
    \subsubsection{Para os ícones utilizamos:}
     as bibliotecas de ícones Phosphor Icons, FlatIcon.
    \subsubsection{Para edição da documentação oficial utilizamos:}
     o LaTeX - LuaLaTex e o Overleaf e TexStudio.
    \subsubsection{Para Designs e Protótipos utilizamos:}
    o Canva e o GIMP.

    
%Metodologia
\subsection{Metododologia de Desenvolvimento}
\subsubsection{Modelo de Desenvolvimento}
Qual metodologia foi utilizada \textbf{Waterfall}.
\subsubsection{Fases do Projeto}
\begin{itemize}
    \item Pesquisa
    \item Levantamento de Requisitos
    \item Diagramação
    \item Prototipaçao
    \item Implementação
    \item Testes
    \item Implantação
    \item Manutenção
\end{itemize}
\subsubsection{Planejamento e Cronograma}
\begin{itemize}
    \item \textbf{Pesquisa (Semana 1-2)}: Nesta fase inicial, foram realizadas pesquisas de mercado e levantamento de referências sobre redes sociais específicas para o nicho imobiliário. Analisamos as funcionalidades e características de plataformas semelhantes para entender as necessidades do público-alvo e as melhores práticas do setor.
    \item \textbf{Levantamento de Requisitos (Semana 3-4)}: Com base na pesquisa, realizamos reuniões com stakeholders para coletar os requisitos funcionais e não funcionais do sistema. Esta fase incluiu a documentação detalhada dos requisitos e a definição clara do escopo do projeto.
    \item \textbf{Diagramação (Semana 5-6)}: Durante esta fase, elaboramos os diagramas UML necessários, incluindo diagramas de casos de uso, diagramas de classes e diagramas de sequência. Esses diagramas ajudaram a visualizar a estrutura e o comportamento do sistema.
    \item \textbf{Prototipação (Semana 7-8)}: Desenvolvemos protótipos das interfaces gráficas para validar as funcionalidades e a usabilidade do sistema junto aos usuários finais. Os feedbacks obtidos foram utilizados para refinamentos antes da implementação.
    \item \textbf{Implementação (Semana 9-14)}: Iniciamos o desenvolvimento do código fonte, dividindo o trabalho entre as equipes de frontend e backend. Foram realizados testes unitários continuamente para garantir a qualidade do código. A integração dos componentes foi realizada ao final desta fase.
    \item \textbf{Testes (Semana 15-16)}: Realizamos testes de integração, testes de sistema e testes de aceitação com usuários. Corrigimos os bugs encontrados e validamos que o sistema atendia aos requisitos especificados.
    \item \textbf{Implantação (Semana 17)}: Preparação do ambiente de produção e implantação do sistema. Treinamos os usuários finais e disponibilizamos a rede social para acesso público.
    \item \textbf{Manutenção (Semana 18 em diante)}: Monitoramento contínuo do sistema, correção de possíveis problemas e realização de atualizações conforme necessário. A manutenção garantiu que o sistema continuasse operando de forma eficiente e segura.
\end{itemize}
% --- Document: End ---
\end{document}

