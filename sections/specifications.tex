% --- Document: Preamble ---
\documentclass[../main.tex]{subfiles}
\graphicspath{{\subfix{../assets/}}}
% --- Document: Start ---
\begin{document}
% Specifications
\section{Arquitetura do Sistema}
% Camadas do Sistema
\subsection{Camadas do Sistema}

O sistema desenvolvido é estruturado em três camadas principais: a camada visual, a camada funcional e a camada de armazenamento. Esta arquitetura em camadas facilita a manutenção e a evolução do sistema, promovendo a separação de responsabilidades e aumentando a modularidade do código.

\subsubsection{Camada Visual}
A camada visual, também conhecida como interface do usuário (UI), é responsável pela interação com o usuário. Ela engloba todos os elementos gráficos e interfaces com as quais o usuário interage diretamente. No nosso projeto, utilizamos Java Swing para criar uma interface intuitiva e amigável, com o auxílio das seguintes tecnologias:
\begin{itemize}
    \item \textbf{Swing + SwingX}: Como biblioteca gráfica.
    \item \textbf{MigLayout}: Biblioteca de layout flexível e responsiva. 
    \item \textbf{Formdev Flatlaf}: Look and Feel para a biblioteca Swing.
    \item \textbf{Formdev IntelliJThemes}: Conjunto de temas inspirados no IntelliJ para a biblioteca Swing.
\end{itemize}
A camada visual captura as ações do usuário e as envia para a camada funcional para processamento.

\subsubsection{Camada Funcional}
A camada funcional, ou lógica de negócios, é onde reside a lógica principal do aplicativo. Esta camada recebe as entradas da camada visual, processa as informações de acordo com as regras de negócio definidas, e retorna os resultados para a camada visual. As tecnologias utilizadas nesta camada incluem:
\begin{itemize}
    \item \textbf{Maven Assembly Plugin}: Plugin Maven para a criação de artefatos executáveis.
    \item \textbf{Maven JavaDoc Plugin}: Plugin Maven para a criação de documentação JavaDoc.
    \item \textbf{BouncyCastle BCProvider}: Biblioteca de criptografia.
\end{itemize}
A camada funcional é essencial para garantir que o sistema opere de acordo com as especificações e requisitos definidos.

\subsubsection{Camada de Armazenamento}
A camada de armazenamento é responsável pela gestão e persistência dos dados utilizados pelo sistema. Embora o projeto não utilize um banco de dados tradicional, implementamos uma solução de armazenamento em memória, utilizando classes de armazenamento (storage classes) para simular um banco de dados. Esta abordagem permite a manipulação eficiente dos dados durante a execução do sistema, proporcionando uma pseudo-persistência necessária para as operações do aplicativo.

O uso de armazenamento em memória simplifica a implementação e é adequado para os objetivos acadêmicos do projeto, facilitando o desenvolvimento e testes sem a complexidade adicional de gerenciar um banco de dados completo.


% Compenentes do Sistema
\subsection{Componentes do Sistema}

Nesta seção, apresentamos os componentes do sistema, que são divididos em diversas áreas funcionais para facilitar o desenvolvimento e manutenção. Cada componente é detalhado a seguir:

\begin{itemize}
    \item \textbf{Estilo: Componentes Básicos (Painéis, Modais e Notificações)}: Desenvolver os componentes básicos de interface, incluindo painéis, modais e notificações.
    \item \textbf{Estilo: Cores e Fontes}: Definir e aplicar a paleta de cores e fontes que serão utilizadas na interface do usuário.
    \item \textbf{Painel: Tela Inicial}: Criar o painel da tela inicial, onde os usuários poderão acessar as principais funcionalidades do sistema.
    \item \textbf{Painel: Criação de Conta - Dados Obrigatórios Básicos}: Implementar a funcionalidade de criação de conta com os dados obrigatórios básicos.
    \item \textbf{Painel: Criação de Conta - Dados Opcionais Extras}: Adicionar campos opcionais extras no processo de criação de conta.
    \item \textbf{Painel: Criação de Conta - Foto de Perfil e Preferências}: Permitir que os usuários adicionem uma foto de perfil e configurem suas preferências durante a criação da conta.
    \item \textbf{Recuperação de Conta: Geração de Código}: Implementar a funcionalidade de recuperação de conta através da geração de um código.
    \item \textbf{Recuperação de Conta: Recuperação de Conta}: Desenvolver o processo de recuperação de conta utilizando o código gerado.
    \item \textbf{Painel de Navegação}: Criar o painel de navegação que permitirá aos usuários acessar diferentes seções do sistema.
    \item \textbf{Feed de Usuário}: Implementar o feed onde os usuários podem ver atualizações e postagens de outros usuários.
    \item \textbf{Propriedades: Criação}: Desenvolver a funcionalidade de criação de propriedades, onde os usuários podem adicionar novas propriedades ao sistema.
    \item \textbf{Propriedades: Edição e Remoção}: Permitir a edição e remoção de propriedades existentes.
    \item \textbf{Postagens: Criação}: Implementar a funcionalidade de criação de postagens pelos usuários.
    \item \textbf{Postagens: Edição e Remoção}: Permitir a edição e remoção de postagens existentes.
    \item \textbf{Postagens: Comentários}: Desenvolver a funcionalidade de comentários em postagens.
    \item \textbf{Perfil: Perfil de Usuário}: Criar a página de perfil do usuário onde informações pessoais e preferências podem ser visualizadas e editadas.
    \item \textbf{Perfil: Preferências de Usuário}: Implementar a funcionalidade de configuração de preferências do usuário.
    \item \textbf{Chat}: Desenvolver a funcionalidade de chat para comunicação entre usuários.
    \item \textbf{Configuração: Escala e Anti-Aliasing}: Implementar configurações de escala e anti-aliasing para melhorar a qualidade visual do sistema.
\end{itemize}

% Tecnologias Utilizadas

\subsection{Tecnologias Utilizadas}
\subsubsection{Editores de Texto}
\begin{itemize}
\item \textbf{Visual Studio Code}

O VS Code é um editor de código-fonte que foi desenvolvido pela Microsoft para os Sistemas Operacionais Windows, Linux e MacOS. Auxilia no suporte de depuração, realce da sintaxe, snippets, refatoração de código, complementação inteligente de código e controle de versionamento Git incorporado
\item \textbf{Neovim}

O Neovim é um editor de textos hiper-extensível baseado em Vim (editor de texto que funcionava via terminal). Ele é uma evolução do Vim e foi criado pelo brasileiro Tiago Padilha. 
\end{itemize}
\subsubsection{Bibliotecas}
\begin{itemize}
    \item \textbf{Maven Assembly Plugin}

    Plugin Maven para a criação de artefatos executáveis.
    \item \textbf{Maven JavaDoc Plugin}

    Plugin Maven para a criação de documentação JavaDoc.
    \item \textbf{Swing + SwingX}

    Interface e Componentes, swing é uma biblioteca gráfica padrão do Java, já o Extensão da biblioteca Swing, fornecendo componentes e funcionalidades adicionais para o desenvolvimento de interfaces gráficas mais ricas e interativas. 
    \item \textbf{MigLayout}

    Biblioteca de layout flexível e responsiva, permitindo a criação de interfaces com posicionamento e redimensionamento precisos dos componentes.
    \item \textbf{Formdev Flatlaf}

    Look and Feel para a biblioteca Swing, oferece uma experiência visual moderna e minimalista, destacando-se por sua estética limpa e elegante, além de proporcionar uma usabilidade intuitiva e responsiva.
    \item \textbf{Formdev IntelliJThemes}

    Conjunto de temas inspirados no IntelliJ para a biblioteca Swing. Oferece uma variedade de opções de temas, permitindo personalizar a aparência da interface gráfica de acordo com as preferências do usuário.
    \item \textbf{BouncyCastle BCProvider }

    Para segurança dos usuários, foi aplicado CRIPTOGRAFIA a partir da biblioteca BouncyCastle com a utilização da classe BCProvider, que atua como um provedor fornecendo implementação de algoritmos criptográficos.
\end{itemize}

% --- Document: End ---
\end{document}

