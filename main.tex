% !TEX TS-program = lualatex
% !TEX encoding = UTF-8 Unicode
% --- Document: Preamble ---
% Formatting
\documentclass[a4paper,12pt]{article}
\usepackage[left=2cm,top=2cm,bottom=2cm,right=2cm]{geometry}

\usepackage{minted}

% Subfiles Support
\usepackage{subfiles}

% Graphics Support
\usepackage{graphicx}
\graphicspath{{./assets/}}

% Font and Language
%\usepackage[scaled]{helvet}
\usepackage{helvet}
\renewcommand{\familydefault}{\sfdefault}
\renewcommand\familydefault{\sfdefault}
\usepackage[T1]{fontenc}
\usepackage[portuguese]{babel}
\addto\extrasportuguese{
    \renewcommand{\contentsname}{\centering{Sumário}}{\vspace{1cm}}
}
\addto\extrasportuguese{
    \renewcommand{\listfigurename}{\centering{Figuras}}{\vspace{1cm}}
}

% Margins, Columns, Indentation & Spacing
\usepackage{multicol}
\usepackage{indentfirst}
\usepackage{ragged2e}  % For extendability
\usepackage{setspace}
\onehalfspacing
\setlength{\columnsep}{1cm}

% Headers, Titles, Footnotes & Captioning
\usepackage{sectsty}
\usepackage{titling}
\renewcommand\maketitlehooka{\null\mbox{}\vfill}
\renewcommand\maketitlehookd{\vfill\null}
\usepackage{footmisc}
\usepackage{caption}
\captionsetup[figure]{font=footnotesize}

% URL Support
\usepackage{hyperref}
\hypersetup{
    colorlinks=true,
    linkcolor=black,
    filecolor=magenta,
    urlcolor=cyan,
    pdftitle={Dream InHouse - Documentação},
    pdfpagemode=FullScreen,
}
\urlstyle{same}

% Pagestyle
\usepackage{fancyhdr}
\pagestyle{fancy}
\fancyhf{}
\fancyfoot[R]{\thepage}
\renewcommand{\headrulewidth}{0pt}

% Document Info
\title{{\Huge Dream InHouse}}
\author{
}
\date{2024}

% --- Document: Start ---
\begin{document}
% Title Page
\begin{titlingpage}
    \begin{center}
        \large
            % Institution
            {\Huge
                FATEC -- Prof. Waldomiro May
            }\\
            Análise e Desenvolvimento de Sistemas

            % Project Logo
            \vspace{3cm}
            \raisebox{-0.75em}{
                \includegraphics[scale=1.25]{logo/main.png}
            }

            % Project Name
            \vspace{1cm}
            {\Huge
                Dream InHouse
            }

            % Project Description
            Rede Social de Nicho Imobiliário

            % Authors
            \vspace{5cm}
            {
                Chrystian Mendes. Franklin\\
                Mariana Guimarães Montoro Morais\\
                Richard Mateus Costa Pinto\\
                Willians Henrique Santos Silva
            }

            % Date
            \vspace{2cm}
            2024
        \normalsize
    \end{center}
\end{titlingpage}
\pagebreak
% Acknowledgements
\begin{centering}
    \section*{Agradecimentos}
    \begin{quotation}
        A Deus, que sempre nos guia os passos e nos dá forças e proteção. 

        À FATEC Prof. Waldomiro May. Ao nosso orientador Professor Eduardo Hidenori Enari, pela competência e carinho com que nos conduziu pelos caminhos da iniciação científica. 
        
        Aos nossos familiares, amigos e a todos aqueles que contribuíram para a realização deste Projeto. 
    \end{quotation}
\end{centering}
\begin{centering}
    \section*{Resumo}
    \begin{quotation}
        A rede social Dream InHouse tem como principal objetivo o nicho imobiliário, dirigida para qualquer pessoa interessada em imóveis, a qual abrange usuários comuns, corretores de imóvel e qualquer pessoa que deseja participar de interações e estudar sobre o mercado imobiliário. A funcionalidade da rede social possui login, tela de login, feed principal, menu lateral, publicação no feed, perfil do usuário, publicações, detalhamento das publicações e notificações. A escolha do nome foi realizada para representar o nicho imobiliário, sendo “Dream InHouse”, utilizado no inglês, sua inspiração vem do português, e trata-se de “Sonho Em Casa”, a ideia de que tal é um projeto ideal para contatos e interações entre pessoas que cultivem interesses em comum, como anunciar, adquirir ou estudar imóveis, sendo assim, os usuários podem realizar o sonho de conquistar um imóvel que corresponda às suas expectativas.

        \textbf{Palavras-chave:} rede social, nicho imobiliário, usuários comuns, corretores de imóvel, mercado imobiliário, contatos, interações, interesses em comum, anunciar, adquirir, estudar imóveis, usuários, pessoas, imóvel
    \end{quotation}
\end{centering}
\begin{centering}
    \section*{Abstract}
    \begin{quotation}
        The Dream InHouse social network has as its main objective the real estate niche, aimed at anyone interested in real estate, which covers ordinary users, realtors and anyone who wants to participate in interactions and study about the real estate market. The social network functionality has login, login screen, main feed, side menu, feed post, user profile, publications, publication details and notifications. The choice of the name was made to represent the real estate niche, being "Dream InHouse", used in English, its inspiration comes from the Portuguese, and it is "Dream At Home", the idea that this is an ideal project for contacts and interactions between people who cultivate common interests, such as advertising, acquiring or studying real estate, so users can fulfill the dream of conquering a property that meets their expectations. 

        \textbf{Keywords:} social network, real estate niche, common users, real estate brokers, real estate market, contacts, interactions, common interests, advertise, acquire, study real estate, users, people, real estate.
    \end{quotation}
\end{centering}
\pagebreak
% Table of Figures
\listoffigures
\pagebreak
% Table of Contents
\tableofcontents
\pagebreak
% Introduction
\subfile{sections/introduction}
\pagebreak
% Cronogram
\subfile{sections/requisitos}
\pagebreak
% Project Specifications
\subfile{sections/specifications}
\pagebreak
% Design
\subfile{sections/implementation}
\pagebreak
% Development
\subfile{sections/tests}
\pagebreak
% Development
\subfile{sections/manutencao}
\pagebreak
% Development
\subfile{sections/conclusao}
\pagebreak
% apendice
\subfile{sections/apendice}
% --- Document: End ---
\end{document}

